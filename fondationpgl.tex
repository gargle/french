\documentclass[11pt, french]{article}

\usepackage[french]{babel}
\selectlanguage{french}
\usepackage[T1]{fontenc}
\usepackage[utf8]{inputenc}
\usepackage{hyperref}
\usepackage{eurosym}

% M-x set-input-method TeX to type œ (\ o e) and Œ (\ O E)

\title{fondationpgl.com}

\begin{document}

\maketitle

\url{https://fondationpgl.ca/accueil/la-dictee-pgl/a-propos/}

\tableofcontents

\vfill

\section{Première année}

\subsection{À l'école}

Voici l'école de Mamadou. Elle est en bois. Elle est belle ! Mamadou aime son école.

\subsection{Bien manger}

Chaque jour, Anne boit huit verres d'eau. Elle mange beaucoup de fruits et de légumes.

\subsubsection{C'est bon !}

Regarde le lion. Il boit de l'eau. Le singe et la souris aussi.

\subsection{En santé}

Nathan est en santé. Chaque jour, il mange des aliments des quatre groupes.

\subsection{Fatou}

Voici Fatou. Dans sa classe, il y a un tableau, un poisson rouge et trois plantes.

\subsection{J'aide maman}

J'adore travailler avec maman. Plus tard, je veux faire son métier. Elle soigne des animaux.

\subsection{L'eau}

L'eau se transforme souvent. Quand la pluie tombe, elle arrive sur moi en forme de goutte.
Quand la neige tombe, c'est un flocon.

\subsection{La fête}

C'est la fête de mon amie. Regarde ! J'ai une belle fleur pour elle.

\subsection{La nature}

Je marche dans la nature avec papa. Je regarde les plantes et les insectes. C'est très joli !

\subsection{La pêche}

Luc pêche avec papa. Il a un poisson dans son sac. Le poisson est pour maman.

\subsection{La peinture}

Je fais de la peinture. Je dessine une fleur jaune et un arbre. J'aime dessiner.

\subsection{Le dessin}

Je dessine un animal. C'est un petit singe bleu. Il est caché sous un éléphant.

\subsection{Le lac}

Je joue dans l'eau avec mon ami. Le lac est bleu. J'aime le lac.

\subsection{Le repas}

C'est ma fête. Je partage un repas avec mon ami Luc. Je mange du poulet. C'est très bon.

\subsection{Mamadou}

Mamadou est heureux. Il va à l'école. Il aime lire et écrire. Vive l'école !

\subsection{Marie}

Voici mon amie Marie. Elle est dans ma classe. Elle aime écrire au tableau.

\section{Deuxième année}

\subsection{À table !}

La mère d'Ariane prépare le repas. Elle prépare du poisson avec des carottes, des oignons et des tomates. Elle fait aussi du riz. Ariane aime manger du poisson. C'est bon pour la santé.

\end{document}
