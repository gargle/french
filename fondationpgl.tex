\documentclass[11pt, french]{article}

\usepackage[french]{babel}
\selectlanguage{french}
\usepackage[T1]{fontenc}
\usepackage[utf8]{inputenc}
\usepackage{hyperref}
\usepackage{eurosym}

% M-x set-input-method TeX to type œ (\ o e) and Œ (\ O E)

\title{fondationpgl.com}

\begin{document}

\maketitle

\url{https://fondationpgl.ca/accueil/la-dictee-pgl/a-propos/}

\tableofcontents

\vfill



\section{Première année}

\subsection{À l'école}

Voici l'école de Mamadou. Elle est en bois. Elle est belle ! Mamadou aime son école.

\subsection{Bien manger}

Chaque jour, Anne boit huit verres d'eau. Elle mange beaucoup de fruits et de légumes.

\subsubsection{C'est bon !}

Regarde le lion. Il boit de l'eau. Le singe et la souris aussi.

\subsection{En santé}

Nathan est en santé. Chaque jour, il mange des aliments des quatre groupes.

\subsection{Fatou}

Voici Fatou. Dans sa classe, il y a un tableau, un poisson rouge et trois plantes.

\subsection{J'aide maman}

J'adore travailler avec maman. Plus tard, je veux faire son métier. Elle soigne des animaux.

\subsection{L'eau}

L'eau se transforme souvent. Quand la pluie tombe, elle arrive sur moi en forme de goutte.
Quand la neige tombe, c'est un flocon.

\subsection{La fête}

C'est la fête de mon amie. Regarde ! J'ai une belle fleur pour elle.

\subsection{La nature}

Je marche dans la nature avec papa. Je regarde les plantes et les insectes. C'est très joli !

\subsection{La pêche}

Luc pêche avec papa. Il a un poisson dans son sac. Le poisson est pour maman.

\subsection{La peinture}

Je fais de la peinture. Je dessine une fleur jaune et un arbre. J'aime dessiner.

\subsection{Le dessin}

Je dessine un animal. C'est un petit singe bleu. Il est caché sous un éléphant.

\subsection{Le lac}

Je joue dans l'eau avec mon ami. Le lac est bleu. J'aime le lac.

\subsection{Le repas}

C'est ma fête. Je partage un repas avec mon ami Luc. Je mange du poulet. C'est très bon.

\subsection{Mamadou}

Mamadou est heureux. Il va à l'école. Il aime lire et écrire. Vive l'école !

\subsection{Marie}

Voici mon amie Marie. Elle est dans ma classe. Elle aime écrire au tableau.

\vfill



\section{Deuxième année}

\subsection{À table !}

La mère d'Ariane prépare le repas. Elle prépare du poisson avec des carottes, des oignons et des tomates. Elle fait aussi du riz. Ariane aime manger du poisson. C'est bon pour la santé.

\subsection{Au restaurant}

Au coin de la rue, il y a un restaurant. C'est le restaurant où Louis travaille. Il fait de la nourriture qui est bonne pour la santé. Tu es d'accord pour y aller ?

\subsection{L'école de Badou}

Je m'appelle Badou. Chaque jour, pour aller à l'école, je marche avec mon amie Noémie. On rit beaucoup.

À mon école, il y a un grand jardin. J'aime bien arroser les plantes.

\subsection{La morue}

Jasmine vit en Haïti. Son papa pêche la morue pour sa famille. La morue vit dans l'eau salée. C'est un poisson de mer.

\subsection{Le griot}

Je vois un griot. Il est sous un arbre. Il raconte une histoire aux enfants. Son histoire parle d'animaux. Les enfants adorent écouter ce conte.

\subsection{Le poète}

Je viens d'un pays très loin. J'écris un poème sur les arbres de mon pays. Mon poème fait rire. Je vais le partager avec tous les enfants de mon école.

\subsection{Le singe}

Regarde le singe ! Il est gris avec une tache jaune. Son nez est rouge. Ses joues sont bleues. Mais où est sa queue ?

\subsection{L'hiver}

L'hiver, les élèves ont hâte de patiner. En février, toute la classe va enfin jouer sur le lac gelé. Les enfants s'habillent très chaudement. Au printemps, la pluie fait fondre la glace et la neige.

\subsection{Mamadou}

Chaque jour, Mamadou boit beaucoup d'eau. L'eau est bonne pour sa santé. Il mange aussi du poulet et des tomates. C'est bon.

\subsection{Mon école}

À mon école, on fait du bricolage, du dessin et de la peinture. On fait aussi des spectacles de théâtre. J'aime mon école.

\subsection{Petite fleur}

Je m'appelle Rose. Ma mère m'appelle Petite Fleur. C'est le petit nom doux qu'elle me donne. J'aime beaucoup ce nom. Il est unique. As-tu un nom doux aussi ?

\subsection{Un bon repas}

Pour son repas, Amadou a apporté du poulet, des légumes et une bonne mangue. C'est un fruit qui pousse dans son pays.

\subsection{Une drôle de façon de manger - Le crocodile}

J'aime le crocodile. Il a une grande bouche et une longue queue. Ses pattes sont courtes et fortes. Il vit en Afrique, comme le lion et la girafe. Son cousin est le caïman.

\subsection{Voice mon ami}

Mon ami Pierre est un garçon unique. C'est mon meilleur ami. À l'école, je travaille toujours avec lui. Nous nous aidons beaucoup.

\vfill

\end{document}
