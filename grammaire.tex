\documentclass[11pt, french]{report}

\usepackage[french]{babel}
\selectlanguage{french}
\usepackage[T1]{fontenc}
\usepackage[utf8]{inputenc}
\usepackage{txfonts}
\usepackage[top=3cm,bottom=3cm,left=3cm,right=3cm,headsep=10pt,a4paper]{geometry} % Page margins



\begin{document}

\part{Tests}

\chapter{Grammaire}

\section{Le nom et l'adjectif}

\subsection{Le genre}

\subsubsection{Complétez les articles ou adjectifs}

\vfill

\begin{enumerate}
  \item As-tu écouté le hit-parade des automobilistes sur Radio 21 ?
  \item << Oh, ce moustique m'agace >>
  \item Après avoir discuté longuement, le Ministre a accordé une interview.
  \item Ce n'est pas un fossé qui se creuse entre nous, c'est un abîme (Duhamel).
  \item Chaque équipe avait du choisir un emblème.
  \item C'est la pire insulte qu'on puisse lui faire
  \item Nul ne peut être heureux s'il ne jouit de sa propre estime (Rousseau).
  \item Voilà un voyou de la pire espèce.
  \item Sur son visage il n'y avait pas le moindre indice de vieillesse.
  \item Il a analysé  le texte suivant une approche typique de Genette.
\end{enumerate}

\vfill

\begin{enumerate}
  \item Faut-il étudier les mots pourvus d'un astérisque ?
  \item Le verbe \textit{avoir}, tout comme le verbe \textit{être}, est un auxiliaire.
  \item Ne crois-tu pas qu'elle aimerait bien gagner le trophée du Belgian Indoor Championship de tennis ?
  \item Quand on la voir courir, on dirait un squelette.
  \item Il a été operé d'un ulcère de l'estomac.
  \item Je me perds toujours dans ce dédale inextricable de ruelles, de carrefours et de culs-de-sac.
  \item Malheureusement, il n'a pas pu venir à cause d'une appendicite.
  \item Comme il croyait avoir trouvé la panacée, il imaginait un monde sans maladies, sans problèmes, sans maux.
  \item Peux-tu réparer la dynamo de ma bicyclette ?
  \item Elle est vraiment le calque de sa mère.
\end{enumerate}

\end{document}
