\documentclass[11pt, french]{report}

\usepackage[french]{babel}
\selectlanguage{french}
\usepackage[T1]{fontenc}
\usepackage[utf8]{inputenc}
\usepackage{txfonts}
\usepackage[top=3cm,bottom=3cm,left=3cm,right=3cm,headsep=10pt,a4paper]{geometry} % Page margins



\begin{document}

\part{Le nom et l'adjectif}

\chapter{Le genre}

\section{Complétez les articles ou adjectifs}

\vfill

\subsection{Formez une phrase}

\begin{enumerate}
\item \textit{Athénée} et \textit{lycée} sont des mots\footnote{mot: m} masculins.
\item Analyséz l'\textit{épisode} le plus passionnant de ce roman.
\item Je voudrais éviter la plus petite \textit{équivoque}\footnote{équivoque: f, dubbelzinnigheid}.
\item Les cris se succédaient à \textit{intervalles} réguliers.
\item Ce n'est là qu'un léger \textit{reproche}\footnote{reproche: m, verwijt} !
\end{enumerate}

\vfill

\begin{enumerate}
\item Dans la salle il y avait une \textit{atmosphère} étouffante.
\item Je me réjouissais en sentant le délicieux \textit{arôme} du café.
\item Il s'est mis à écrire une \textit{satire} violente contre ses adversaires\footnote{adversaire: m, tegenstrever}.
\item Le premier \textit{satellite} a été lancé en 1957.
\item Une \textit{épithète}\footnote{épithète: f, bijnaam} est généralement un adjectif qualificatif.
\end{enumerate}

\vfill

\subsection{Complétez les articles ou adjectifs}

\begin{enumerate}
\item As-tu écouté le \textit{hit-parade} des automobilistes sur Radio 21 ?
\item << Oh, ce \textit{moustique} m'agace >>
\item Après avoir discuté longuement, le Ministre a accordé une \textit{interview}.
\item Ce n'est pas un fossé qui se creuse entre nous, c'est un \textit{abîme} (Duhamel).
\item Chaque équipe avait du choisir un \textit{emblème}.
\item C'est la pire \textit{insulte} qu'on puisse lui faire
\item Nul ne peut être heureux s'il ne jouit de sa propre \textit{estime}\footnote{estime: f, respect} (Rousseau).
\item Voilà un voyou de la pire \textit{espèce}.
\item Sur son visage il n'y avait pas le moindre \textit{indice} de vieillesse.
\item Il a analysé  le texte suivant une \textit{approche} typique de Genette.
\end{enumerate}

\vfill

\begin{enumerate}
\item Faut-il étudier les mots pourvus d'un \textit{astérisque} ?
\item Le verbe \textit{avoir}, tout comme le verbe \textit{être}, est un \textit{auxiliaire}.
\item Ne crois-tu pas qu'elle aimerait bien gagner le \textit{trophée} du Belgian Indoor Championship de tennis\footnote{tennis: m} ?
\item Quand on la voir courir, on dirait un \textit{squelette}.
\item Il a été operé d'un \textit{ulcère} de l'estomac\footnote{estomac: m}.
\item Je me perds toujours dans ce \textit{dédale}\footnote{dédale, m: doolhof} inextricable\footnote{inextricable: onoplosbaar} de ruelles\footnote{ruelle: f}, de carrefours\footnote{carrefour: m} et de culs-de-sac\footnote{cul-de-sac: m}.
\item Malheureusement, il n'a pas pu venir à cause d'une \textit{appendicite}.
\item Comme il croyait avoir trouvé la \textit{panacée}, il imaginait un monde sans maladies\footnote{maladie: f}, sans problèmes\footnote{problème: m}, sans maux\footnote{mal: m}.
\item Peux-tu réparer la \textit{dynamo} de ma bicyclette ?
\item Elle est vraiment le \textit{calque} de sa mère.
\end{enumerate}

\end{document}
