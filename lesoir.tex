\documentclass[11pt, french]{article}

\usepackage[french]{babel}
\selectlanguage{french}
\usepackage[T1]{fontenc}
\usepackage[utf8]{inputenc}
\usepackage{lmodern}
\usepackage{hyperref}
\usepackage{eurosym}

% M-x set-input-method TeX to type œ (\ o e) and Œ (\ O E)
% M-x auto-fill-mode

\begin{document}

\vfill

\section{La protection des données~: mesures à prendre~!}

Une nouvelle législation européenne sur la protection des données
entre bientôt en vigueur. Le secteur culturel et le marché de l'art
doivent également s'y préparer au plus vite~: analyse

La date circule depuis plusieurs mois déjà~: une nouvelle
réglementation sur la protection des données personnelles entrera en
vigueur le 25 mai 2018 dans toute l'Union européenne, plus connue sous
le sigle <<~~RGPD\footnote{Pour Règlement général sur la protection des
  données personnelles, lui-même faisant référence au règlement (UE)
  2016/679 du Parlement européen et du Conseil du 27 avril 2016
  relatif à la protection des personnes physiques à l'égard du
  traitement des données à caractère personnel et à la libre
  circulation de ces données.}~>>.

Tout professionnel est supposé intégrer dans ses activités économiques
la réalité de cette nouvelle réglementation alors que les acteurs de
la culture et du marché de l'art laissent penser à une certaine
inertie en la matière\ldots\ Comme si ce règlement ne les concernait
pas.

Pourtant, tous le professionnels actifs dans le secteur (artistes,
maisons de vente, galeries, éditeurs, industries culturelles et
créatives\ldots) sont concernés ! Tous devraient prendre des mesures
qui s'imposent dans les plus brefs délais. Voici pourquoi.

\subsection{Champ d'application}

Tout comme les autres secteurs, le secteur culturel est lui aussi
concerné par le RGPD car son champ d'application est tellement large
que tout professionnel est susceptible d'y être confronté. Peu importe
que l'on soit une petite ASBL culturelle ou une maison de vente
internationale. Et pour cause~: le champ d'application vise tout
traitement de données à caractère personnel, automatisé en tout ou en
partie, ainsi que le traitement non automatisé de données à caractère
personnel contenues ou appelées à figurer dans un fichier. Les données
à caractère personnel portent sur toutes les informations se
rapportant à une personne physique identifiée ou identifiable~: nom,
prénom(s), coordonnées de contact, identité culturelle, centres
d'intérêt éventuels\ldots

Les exemples de collecte, de conservation ou de traitement de données
personnelles sont infinis. Une mailing liste d'une galerie ou d'un
artiste réservée aux invitations à un vernissage constitue par example
déjà un traitement de données.

\subsection{Urgence~: se conformer au RGPD}

L'acteur culturel qui souhaite (ou qui doit) collecter des données
dans le cadre de ses activités devra procéder par ordre pour respecter
ce règlement.

La première obligation porte sur l'acceptation (et a fortiori son
information) de la personne concernée quant au fait que ses données
sont collectées. Son accord doit être éclairé en ce sens où il n'est
pas possible de préremplir le formulaire ou de présumer qu'elle a
donné un accord. A ce titre, elle doit également savoir ce qui sera
fait de ses données (par exemple~: seront-elles transmises à des
tiers~?), pourquoi ces données spécifiques sont nécessaires ou encore la
durée de la conservation.

La personne concernée doit ensuite pouvoir accéder librement à ses
données, les rectifier et s'opposer à une utilisation.

A certaines conditions, le RGPD prévoit également le droit à l'oubli
de la personne concernée, à savoir le droit d'obtenir du responsable
du traitement l'effacement, dans les meilleurs délais, de données à
caractère personnel la concernant.

Un mécanisme d'information doit être structuré par le professionnel
lorsqu'une violation de donnée est constatée.

Enfin, citons encore l'obligation, dans certaines hypothèses, de
désigner au sein de la structure un délégué à la protection des
données, lequel doit avoir reçu une formation spécifique sur le sujet.

Pensez <<~conditions générales~>>

Une manière d'informer les personnes concernées quant à leurs droits
découlant du RGPD est de les reprendre dans les conditions générales
de la structure culturelle. C'est donc le premier document qui
nécessitera d'être adapté à cette nouvelle législation dont nous
allons tous bénéficier\ldots\ en tant que fournisseur de données.

A bon entendeur\ldots

\end{document}
