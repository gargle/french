\documentclass[11pt, french]{report}

\usepackage[french]{babel}
\selectlanguage{french}
\usepackage[T1]{fontenc}
\usepackage[utf8]{inputenc}
\usepackage{textcomp}

\begin{document}

\chapter{29 décembre 2014 : Les bons réflexes en cas de froid}

\textbf{Alors que les activations de plans \og{}Grand Froid \fg{} se succèdent dans
  les départements, quelques précautions simples permettent de prévenir les
risques pour la santé.}

\vfill

Les basses températures peuvent s'avérer dommegeables pour la sante, surtout pour les personnes sensibles - personnes âgées, nourrissons, malades chroniques, personnes à mobilité réduite, sans domicile, ou encore travailleurs exposés au froid.

Le ministère de la Santé invite par conséquent à redoubler de vigilance face aux effets souvent sous-estimés des températures négatives qui demandent des efforts supplémentaires à l'organisme, notament au cœur, qui bat plus vite pour éviter que le corps ne se refroidisse. Elles peuvent entraîner des hypothermies - quand la température corporelle descend en-dessous de 35\textdegree C - et des gelures: exposées au froid, les extrémités du corps peuvent devenir d'abord rouges et douloureuses, plus grises et indolores.

\begin{itemize}
\item Conseils à l'extérieur:
  \begin{itemize}
  \item Limitez les sorties si vous faites partie des personnes à risque, notamment le soir où les températures perdent plusieurs degrés.
  \item Si vous devez vous déplacer, habillez-vous chaudement, en couvrant bien les parties du corps qui laissent s'échapper la chaleur: la tête en particulier, mais également le cou, les mains et les pieds. Mettez plusieurs couches de vêtements (au moins trois), plus efficaces qu'une seule couche épaisse, et terminez par un manteau coupe-vent et imperméable.
  \item Maintenez un certain niveau d'exercice régulier tel que la marche, sans pour autant faire des efforts trop importants.
  \item Prêtez une attention particulière à votre alimentation, notamment au petit déjeuner. Consommez en priorité des sucres lents (pâtes, pain, riz, lentilles, céréales) et des protéines. Contrairement aux idées reçues, il n'est pas nécessaire de manger plus gras que d'ordinaire.
  \item Buvez beaucoup d'eau, même si vous ne ressentez pas la soif, car le froid et certains chauffages déshydratent. Limitez votre consommation d'alcool, qui ne réchauffe pas !
  \item Évitez les longs trajets en voiture. Si c'est impossible, munissez-vous d'eau, de nourriture, de couvertures, d'un téléphone chargé et renseignez-vous sur la météo avant de partir.
  \item  Si vous devez emmener un enfant, ne le placez pas dans un porte-bébé, portez le dans vos bras ou choisissez une poussette pour que le bébé puisse remuer ses membres et se réchauffer.
  \end{itemize}
\end{itemize}

\begin{itemize}
\item Conseils à la maison:
  \begin{itemize}
  \item Maintenez la température ambiante à un niveau convenable, y compris dans la chambre à coucher (minimum de 19\textdegree C).
  \item Assurez-vous du bon fonctionnement des appareils de chauffage et de leur entretien auprès d'un professionnel avant de les utiliser. Ne surchauffez pas les poêles à bois ni les chauffages d'appoint à cause des risques d'incendie et d'intoxication au monoxyde de carbone.
  \item N'obstruez pas les bouches d'aération, qui permettent d'assurer votre sécurité et de renouveler l'air sans exposer au froid.
  \item Pensez à prendre régulièrement des nouvelles des personnes âgées ou handicapées isolées de votre entourage.
  \end{itemize}
\end{itemize}



\end{document}
