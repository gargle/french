\documentclass[11pt, french]{report}

\usepackage[french]{babel}
\selectlanguage{french}
\usepackage[T1]{fontenc}
\usepackage[utf8]{inputenc}
\usepackage{hyperref}
\usepackage{eurosym}



\begin{document}

\chapter{21 avril 2016 : Comme par enchantement}

\url{https://francebienvenue2.wordpress.com/2016/04/21/comme-par-enchantement/}

\vfill

En fin de compte, Paris est très proche de Marseille : en TGV, il faut environ
trois heures de gare à gare, c'est-à-dire du centre-ville de Marseille à Paris
intra-muros. Et pas besoin d'arriver des heures à l'avance : on peut monter
dans le TGV au dernier moment (entre deux et cinq minutes juste avant que les
portes se ferment).  Pas besoin non plus d'enregistrer ses bagages : ils
voyagent dans le même wagon que vous. Pas besoin de se ruiner non plus : en
s'y prenant bien -- traduisez par à l'avance -- on trouve des tarifs
intéressant.

Alors, pour concurrencer le train, les avions doivent offrir la même sensation
de facilité : c’est ce que nous vend cette publicité rencontrée hier dans la
rue sur un abri-bus.

On ne monte plus dans un avion pour partir en voyage, on prend une navette,
c’est-à-dire un moyen de transport quotidien qui fait la navette entre le nord
et le sud. Et bien sûr, il y a plusieurs vols allers-retours par jour, à des
heures pratiques pour partir et revenir.

Donc, hop, on part pour Paris ! Vous êtes à Marseille, et d’un coup de baguette
magique\footnote{\textbf{d’un coup de baguette magique} : L’expression
  « d’un coup de baguette magique / en un coup de baguette magique » signifie
  que quelque chose se fait très facilement, sans effort.
  Par exemple, on dit : ça ne se fera pas d’un coup de baguette de magique.
  Tu crois que tu vas réussir à trouver un appartement comme ça, en un coup de
  baguette magique ? }\ldots\ Pardon, d’un coup de navette magique, hop, en
une heure, vous voilà à Paris ! ( Enfin presque, parce qu’il faut aller à
l’aéroport, arriver en avance. Et côté prix, ce n’est pas toujours aussi
magique que les 49\euro\ annoncés : tout est dans le « à partir de\ldots\ ». Et
parce que vous atterrissez à Orly, bien relié à Paris, mais quand même. )

En tout cas, la pub est jolie, avec son jeu de mots basé sur les sonorités
proches de navette et de baguette. Un bon slogan, qui met un peu de féérie dans
notre quotidien où parfois, au milieu de la frénésie des transports pour aller
à droite, à gauche, on se prend à rêver de télétransportation ! Vive les
navettes magiques ! 

\vfill

\end{document}

