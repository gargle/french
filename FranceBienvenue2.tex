\documentclass[11pt, french]{report}

\usepackage[french]{babel}
\selectlanguage{french}
\usepackage[T1]{fontenc}
\usepackage[utf8]{inputenc}
\usepackage{hyperref}
\usepackage{eurosym}

% M-x set-input-method TeX to type œ (\ o e) and Œ (\ O E)

\begin{document}

\chapter{11 octobre 2017 : Auto-portrait dans l'auto}

\vfill

Je suis tombée sur cette campagne de prévention sur le danger des
selfies au volant.

C'est l'occasion de se retrouver une fois de plus confronté à la force
de cet appareil qui a envahi nos vies au point qu'il faille diffuser
ce genre de message. Et bien sûr, c'est aussi l'occasion de parler du
français et de la prononciation de note langue. Et donc de remettre en
route mon blog, quelque peu délaissé ces derniers mois ! Etes-vous
toujours là ? J'espère !

Ceux qui ont rédigé ce message jouent avec les mots, pour lui donner
la force d'un slogan facile à retenir : \textbf{L'auto-portrait, oui. Dans
  l'auto, non.}

J'avais presque oublié le mot \textit{auto-portrait}, à force
d'entendre le mot \textit{selfie} en permanence. C'est vrai que
\textit{selfie} ne désigne que les auto-portraits pris avec un
téléphone portable.

\vfill

\chapter{14 juin 2017 : Brindezingues !}

\url{https://francebienvenue2.wordpress.com/2017/06/14/brindezingues/}

\vfill

Mon coup de cœur de ces dernières semaines ! Et je vais garder ce livre précieusement, pour ne
plus quitter Nathalie et Eugène, les dexu enfants / adolescents imaginés par Véronique Ovaldé
et dessinés par Joann Sfar. C'est ne pas une BD, mais une histoire qui naît des mots vivants,
poétiques et drôles de l'écrivaine et des illustrations imaginées à partir du texte par le
dessinateur.

\chapter{15 juin 2016 : Ah bon ? Y a foot ce soir ?}

\url{https://francebienvenue2.wordpress.com/2016/06/15/ah-bon-y-a-foot-ce-soir/}

\vfill

A Marseille, le foot, on connaît. Difficile d’y échapper. Et on s’en est bien
rendu compte avec les violences des hooligans déchaînés ce weekend au centre-ville.
Une chose est sûre, il est partout et sert à tout, même là où il paraît le plus
incongru.

Et comme le foot et la gastronomie ne vont pas tout à fait ensemble et qu’il faut
se nourrir – à défaut de boire – pendant les matches, les marques de surgelés
n’allaient pas laisser passer l’occasion.

\textbf{Donc voici une pub reçue par mail, avec, comme souvent, jeux de mots, expressions,
  formules. Bref du français en action. C’est le bon côté des pubs !}

C’est foot, alors restons dans le registre du foot : \textbf{à l’heure du coup d’envoi du
  match}, il faut être prêt devant sa télé. Donc c’est aussi l’heure du coup
d’envoi – un peu avant – pour les commandes par internet de bonnes pizzas
surgelées.

\textbf{Les enfants regardent aussi} (davantage de garçons que de filles en France) .
Ambiance familiale. Tout le monde vit à l’heure du foot. (Il paraît qu’on est
parti pour une cinquantaine de matches, en un mois\ldots\ )

Alors, voici des glaces pour \textbf{les enfants, qui ne resteront pas sur la touche},
c’est-à-dire pour nos petits qu’on n’oublie pas, qu’on fait participer à
l’événement. Rester sur la touche, c’est regarder ses coéquipiers courir sur le
terrain lorsqu’on est le joueur puni pour faute ou le joueur remplaçant qui attend
son heure sur le banc de touche. Des jolies glaces comme des ballons de foot bien
sûr.

\textbf{Mais le foot s’est glissé dans un tout autre univers, sur un compte instagram
  qu’on pouvait penser imperméable à cet événement sportif !} Humour, second degré,
téléscopage de deux mondes.

Mais aussi téléscopage des styles : je n’aurais pas pensé à employer le
qualificatif de « beaux mecs » à propos des sculptures de Rodin ( et pas forcément
non plus à propos des footballeurs!) , même s’il a effectivement travaillé
certains de ses modèles comme des athlètes. Terme un peu trop familier ! Tout comme
le ton, oral, du « Y’a » qui nous interpelle au début de la phrase. Humour décalé,
pour « dépoussiérer » les musées ?

\textbf{A ce propos, je ne sais pas pourquoi on trouve très souvent écrit : Y’a},
avec cette apostrophe qui n’a pas de sens, et qui tend probablement à restituer le
côté oral de « Il y a » prononcé ainsi quand on parle de façon familière. L’apostrophe
sert en général à indiquer qu’on a supprimé une lettre, comme par exemple « que » qui
devient qu’. Mais ici, il ne manque rien entre Y et a.

\textbf{Bon, en attendant, y a 80 minutes que le match du jour est commencé et les
  beaux mecs de l’équipe de France ne brillent pas en face des beaux mecs de l’équipe
  albanaise au stade Vélodrome de Marseille ! A l’heure où j’écris, y a toujours 0-0.
  Mais qu’est-ce qu’ils font, tous ces beaux mecs avec leur ballon rond ?}

\vfill

\chapter{21 avril 2016 : Comme par enchantement}

\url{https://francebienvenue2.wordpress.com/2016/04/21/comme-par-enchantement/}

\vfill

En fin de compte, Paris est très proche de Marseille : en TGV, il faut environ
trois heures de gare à gare, c'est-à-dire du centre-ville de Marseille à Paris
intra-muros. Et pas besoin d'arriver des heures à l'avance : on peut monter
dans le TGV au dernier moment (entre deux et cinq minutes juste avant que les
portes se ferment).  Pas besoin non plus d'enregistrer ses bagages : ils
voyagent dans le même wagon que vous. Pas besoin de se ruiner non plus : en
s'y prenant bien -- traduisez par à l'avance -- on trouve des tarifs
intéressant.

Alors, pour concurrencer le train, les avions doivent offrir la même sensation
de facilité : c’est ce que nous vend cette publicité rencontrée hier dans la
rue sur un abri-bus.

On ne monte plus dans un avion pour partir en voyage, on prend une navette,
c’est-à-dire un moyen de transport quotidien qui fait la navette entre le nord
et le sud. Et bien sûr, il y a plusieurs vols allers-retours par jour, à des
heures pratiques pour partir et revenir.

Donc, hop, on part pour Paris ! Vous êtes à Marseille, et d’un coup de baguette
magique\footnote{\textbf{d’un coup de baguette magique} : L’expression
  « d’un coup de baguette magique / en un coup de baguette magique » signifie
  que quelque chose se fait très facilement, sans effort.
  Par exemple, on dit : ça ne se fera pas d’un coup de baguette de magique.
  Tu crois que tu vas réussir à trouver un appartement comme ça, en un coup de
  baguette magique ? }\ldots\ Pardon, d’un coup de navette magique, hop, en
une heure, vous voilà à Paris ! ( Enfin presque, parce qu’il faut aller à
l’aéroport, arriver en avance. Et côté prix, ce n’est pas toujours aussi
magique que les 49\euro\ annoncés : tout est dans le « à partir de\ldots\ ». Et
parce que vous atterrissez à Orly, bien relié à Paris, mais quand même. )

En tout cas, la pub est jolie, avec son jeu de mots basé sur les sonorités
proches de navette et de baguette. Un bon slogan, qui met un peu de féérie dans
notre quotidien où parfois, au milieu de la frénésie des transports pour aller
à droite, à gauche, on se prend à rêver de télétransportation ! Vive les
navettes magiques ! 

\vfill

\chapter{17 avril 2016 : Le jeune papa qui voulait apprendre à cuisiner}

\url{https://francebienvenue2.wordpress.com/2016/04/17/le-jeune-papa-qui-voulait-apprendre-a-cuisiner/}

\vfill

C’est très à la mode de raconter des histoires de la vie des gens ordinaires.
C’est ce qu’ont compris les entreprises pour se faire de la publicité, poussant
ainsi plus loin la vieille idée de mettre en scène des utilisateurs de leurs
produits. Maintenant, ce sont de véritables petits films, à regarder sur leur
site ou les réseaux sociaux. Et la cuisine est partout ! Je suis donc tombée sur
de nouvelles vidéos qui mettent en scène un cuisinier plein d’énergie et des
apprentis-cuisiniers, avec en arrière-plan cette fois-ci une marque d’huile
qu’on trouve partout. (Mais ils ne passent pas leur temps à mentionner le nom
de cette huile. Donc c’est regardable comme une petite tranche de vie.)

Encore de la cuisine, allez-vous dire ! Oui, mais c’est surtout parce que j’ai
trouvé que les échanges entre Fred et ceux qui ont besoin de ses conseils étaient
une belle leçon d’oral, pleine de naturel, de mots familiers, la vraie vie donc !
Et accessoirement, j’ai retenu une astuce pour faire une excellente mayonnaise,
que je testerai un de ces jours !

\vfill

Vous savez quoi ? Pour moi, la cuisine populaire, c’est à la fois des plats
simples et gourmands. Bah ça, c’est tout ce que j’adore. Mais cuisiner au
quotidien, c’est parfois compliqué pour ceux qui manquent de temps, d’expérience
ou encore d’idées. C’est pourquoi j’ai décidé d’aller à leur rencontre, pour
partager avec eux toutes les petites astuces que j’ai pu glaner au cours de ma
vie de cuistot\footnote{\textbf{un cuistot} : abréviation familière de
  \textit{cuisinier}.} tout terrain\footnote{\textbf{tout terrain} : emloyée ici,
  cette expression indique qu’il sait cuisiner en toutes circonstances, en
  s’adaptant à tout. Normalement, ce sont les vélos ou les véhicules qui sont
  tout terrain.}. Parce que après tout, cuisiner sans en faire tout un plat
\footnote{\textbf{sans en faire tout un plat} : il joue sur les mots puisqu’on
  est dans le domaine de la cuisine. Cette expression signifie qu’on n’accorde
  pas plus d’importance que nécessaire à quelque chose, qu’on ne complique pas
  la situation inutilement.}, croyez-moi, c’est possible.

-- Oui, oui, non non, t’inquiète pas, non, non, je t’attends, non, non. Je suis
juste à la sortie du métro. Non, non, prends ton temps. Tout va bien,
à toute\footnote{\textbf{À toute} : c’est le raccourci de
  \textit{A tout à l’heure}, qui signifie qu’on se voit très bientôt dans la
  suite de la journée.}, ciao.

Là, j’étais en ligne avec Cédric. Il est toujours à la
bourre\footnote{\textbf{Être à la bourre} : être en retard (très familier)}.
Ça, on m’avait prévenu. En fait, il est éditeur de jeux vidéo, il bosse comme
un dingue\footnote{\textbf{bosser comme un dingue} : travailler comme un
  fou (familier)}, et en plus de ça, il vient d’avoir un petit bébé avec sa
compagne\footnote{\textbf{sa compagne} : c’est le terme qu’on emploie quand
  dans un couple, les gens ne sont pas mariés. Pour un homme, on utilise le
  masculin : son compagnon. \textit{Cédric est le compagnon de Malène}.} Malène.
Alors, jusqu’à présent, c’était elle qui était la préposée aux fourneaux
\footnote{\textbf{être préposé aux fourneaux} : c’est l’expression habituelle
  pour parler de celui qui a pour tâche de faire la cuisine.}. Mais là, depuis
l’arrivée du petit, franchement, il a décidé de mettre la main à la pâte
\footnote{\textbf{Mettre la main à la pâte} : encore une expression basée sur
  la cuisine, qui signifie qu’on apporte son aide, qu’on contribue à faire
  quelque chose. (familier)}.

-- Enchanté. Ça va ?

-- Désolé, je suis à la bourre.

-- Non, mais t’inquiète\footnote{\textbf{T'inquiète} : Ne t’inquiète pas.
  Cette fois, les deux termes de la négation ont disparu. (Très familier)},
tout va bien !

-- Donc j’ai pas eu du tout le temps de faire les courses.

-- Eh bah super ! Comme ça, ça veut dire qu’on va les faire !

-- Voilà. Nickel\footnote{\textbf{Nickel} : parfait (familier). Ce terme a
  tendance à remplacer Super, notamment chez lez jeunes.}.

-- Je te suis parce que moi, je suis en terre inconnue.

-- Tu es en terre inconnue ! Tu es à Montreuil ! Bienvenue à Montreuil.

-- Bonjour monsieur.

-- Bonjour.

-- Ah, j’adore ce genre d’épicerie !

-- On a beaucoup voyagé. On est resté deux semaines en Thaïlande, Malène a
pris des cours de cuisine thaï, mais pas moi. Donc Fred…

-- OK, donc je vais être ton coach de cuisine thaï,si je comprends bien. Tu sais
en Thaïlande, tu as dû sûrement manger… il y a deux plats nationaux, c’est le
Tom Kha Kai, la soupe de poulet au galanga et à la citronnelle. Et puis tu as le
fameux curry vert. Moi, je suis le roi du curry vert.

-- Ça, tu vas nous faire plaisir ! On adore !

-- Les piments.

-- Les fameux petits piments oiseau. Ciboule. Ah, des petites tomates ! Gingembre
et citronnelle. Quatre beaux citrons verts. Vous avez du basilic thaï ? Ouais,
super. C’est à la fois un compromis de basilic tel que tu le connais mais anisé
et citronné.

-- Anis, c’est ça.

-- Merci messieurs. Allez au revoir.

-- C’est nous !

-- Bonjour.

-- Bonjour.

-- Malène.

-- Oui, enchanté.

-- Bonjour. Enchantée.

-- C’est quoi, ce charmant petit accent ?

-- Ça vient du Danemark.

-- Non ! Tu es danoise ! Ah, j’adore ! C’est toi, Malène, qui cuisines, hein.

-- Ouais. Enfin à 95 \%.

-- Tu sais quoi ? Bientôt, ça va être Cédric.

-- Ah, j’ai hâte !

-- Je jette un petit coup d’oeil dans le frigo. Ah oui, d’accord ! Ça, c’est
du frigo (12) bien rempli, hein ! Si on fait duo petites pétoncles et crevettes,
c’est parti !

-- Bon bah, je vous laisse travailler, les garçons, et je vous dis à plus tard.

-- Bon alors, l’idée, c’est qu’on va réaliser une pâte de curry. J’ai mis
l’équivalent de deux petites bottes de coriandre (13). On va mettre dix brins
de ciboule. Tout simplement, tu vas l’ébarber. Hop ! Ensuite, la citronnelle,
tu coupes la base, une petite encoche, comme ceci. Bravo.

-- Et là, j’enlève…

-- Là, tu vas l’éplucher.

-- Et donc là, donc faut que je coupe.

-- Tu vois ce mouvement ?

-- Alors, je descends et après, je pousse.

-- Tu pousses, tu casses ton poignet. Clac !

-- Ah oui carrément, c’est technique, hein !

-- Super, parfait. Tu l’as, le geste. C’est parfait, c’est nickel. Je le mets
à nouveau directement dans la cuve du robot. Ensuite, l’ail.

-- Ça, je crois que je sais faire.

-- Vas-y. Montre-moi.

-- C’est un truc comme ça, je crois.

-- Attends, attends ! Ouh là, ouh là, là ! Qu’est-ce que… Ouh là, là là !
Oh là, là ! Oh, ça sent les urgences (14)! Oh mon dieu, tu m’as fait peur !
Le tranchant toujours vers l’extérieur. Et là, tu peux… Tac ! Voilà. Parfait.
Alors, ensuite, le gingembre. Alors, il y a des fibres partout mais d’autant
plus au centre.

-- C’est comme une mangue finalement, non ?

-- Exactement. Ce qui nous intéresse, c’est juste ce qu’il y a pile en-dessous
la peau. Une fois que tu l’as bien épluché, tu vas le râper. Mais attention, tu
vas t’arrêter pile juste là (15), à l’arrivée des grosses fibres. Il nous faut
en fait l’équivalent d’une cuillère à soupe bombée de gingembre. Hop ! Alors
ensuite, tiens, notre fameux petit piment. Comme Malène est en train d’allaiter,
un seul, ça suffira. Je vais le fendre en deux. En fait, tu sais, il faut
absolument enlever les petites graines. Et puis tu sais, les petites graines
blanches, là tu as ici le centre en théorie d’un cercle. Tu poses ta main ici,
et ensuite, tu vas rayonner. Bravo, parfait ! Tu maîtrises ! Ensuite, tu mets le
zeste d’un seul citron. Si tu râpes un peu trop, tu vas attaquer le blanc et le
blanc, c’est très amer. Tu frôles la petite râpe comme ceci. Notre fameux basilic
thaï, on va effeuiller à peu près la moitié de cette botte, sachant que l’autre
moitié, on va la réserver pour la fin de la cuisson. Tout ça, on va le mettre sur
ton super joli robot. Et là, on va rajouter de l’huile, fleur de colza, c’est
nickel, jusqu’à temps qu’on (16) obtienne la consistance d’un pesto. Voilà, là,
c’est bien souple. Sens-moi ça !

-- Oh, c’est top !

-- Hein, on y est, hein ! Ça c’est une pâte de curry. Sache (17) une chose, c’est
que cette pâte-là, elle a le mérite et l’avantage de se congeler.

-- On peut en faire beaucoup plus et…

-- Eh bah c’est ça, tu vas doubler, tu vas quadrupler les quantités. Après, tu
fractionnes ça dans des sacs congélation. Au dernier moment, imagine que tes
invités arrivent dix minutes après, au débotté (18), tu peux les épater (19).
Maintenant que je sais que Malène est danoise, j’ai un peu reluqué (20) ce qu’il
y avait dans ton frigo. J’ai vu qu’il traînait du saumon et j’ai vu surtout un
truc génial, ce petit reste de pommes de terre, là. Je vais t’apprendre la petite
sauce danoise qui va parfaitement avec la salade de pommes de terre. Là, on va
faire une mayonnaise vraiment pas comme les autres. Alors maintenant, si tu as
pas le temps de faire une mayonnaise, rien ne t’empêche de prendre une mayonnaise
déjà toute faite, il y a zéro stress. En fait, cette petite mayonnaise, elle est
extraordinaire, tout simplement à cause d’un seul truc, c’est qu’elle est faite
avec un œuf, mais attention, un œuf coque (21).

-- Ah oui !

-- Ça bout (22). Et là, voilà. Un petit trois, trois quarante.

-- Je coupe en morceaux.

-- En petites rondelles, voilà comme ça. Là, voilà, on va casser la cuisson. Là, je
vais l’écaler (23). Je vais le mettre dans le fond. Ensuite, tu mets l’équivalent
d’une cuillerée à soupe de moutarde. Je vois que tu as une huile tout terrain,
Isio 4 (24), tu vas mettre un petit 15 centilitres d’huile, et là, tu vois,
l’émulsion prend tout de suite. Regarde, elle est crémeuse. Ça, c’est l’effet œuf
coque. Ta mayonnaise, elle a un goût d’oeuf coque.

-- Une mayonnaise comme ça, je la ferai.

-- On va alléger cette mayonnaise. Tu sais, tu prends un petit yaourt bulgare, un
truc comme ça, là, comme tu as. C’est là où le fameux curry de madras rentre en
scène.

-- Deux, trois pincées, quoi.

-- Exactement. On va mélanger. J’ai vu que dans ton frigo, tu avais une petite
botte de ciboulette qui traînait. Voilà. C’est parfait, c’est nickel.

-- Ça sent bon !

-- Allez, viens là, Malène. Je voulais te faire une petite surprise. J’ai fait
une salade de pommes de terre à la danoise !

-- Comment on dit en danois ?

-- Hum ! Bon appétit !

-- Il y a du curry dedans. Mayonnaise.

-- Mayonnaise maison.

-- Mayonnaise maison en plus !

-- Avec une technique incroyable d’ailleurs ! Mais je te la donne pas ! C’est la
mienne, hein.

-- Ah, elle est excellente ! Ouah, meilleure que celle de ma mère ! J’ai
l’impression d’être chez moi, là, un peu au Danemark.

-- Ah, ça me fait plaisir, ça ! Deuxième round, c’est parti. On est d’accord que
l’huile est déjà incorporée dans la pâte de curry. Ça, je vais en réserver un tout
petit peu pour la fin. On va faire saisir ça sur feu vif, hein, pendant une minute.
Le lait de coco.

-- Je coupe les tomates en deux ?

-- Tu me coupes les tomates en deux. Moi, pendant ce temps-là, je vais écosser (25)
les petits pois.

-- Mon père, il a un jardin potager. Et mon grand-père avait un jardin potager. Tu
vois, récupérer les petits pois…

-- Ecosser les petits pois ?

-- Ça c’est des souvenirs d’enfant quand tu es gamin!

-- Une punition ?

-- Cédric, viens m’aider !

-- Ok J’arrive.

-- Ensuite, crevettes.

-- Crevettes direct.

-- Au bout de deux minutes, je rajoute les petites pétoncles, comme ceci.

-- Ah l’odeur !

-- Tu as vu ! C’est quelque chose (26), hein ! Et là, au dernier moment, tu
rajoutes ça. Ça va non seulement lui redonner de la couleur, mais en plus,
paf (27), tout le côté herbacé, tu vas l’avoir en bouche. Je mélange une dernière
fois. Et là, citron vert. Tout le bénéfice des citrons verts, c’est qu’il y a pas
de pépins. Donc tu peux y aller comme ça, sans aucun problème. Là, je vais
rajouter la sauce poisson, le fameux Nuocmam. Est-ce que tu as un petit peu de
sucre par hasard ? Ça, on va en mettre un tout petit peu pour casser l’acidité du
citron vert et du nuocmam. Et là, on n’oublie pas, tu te souviens, on avait
réservé un peu de basilic thaï. Voilà.

-- Et tu le mets entier comme ça ?

-- Direct comme ça, au tout dernier moment. Là, je peux te dire que Cédric a pas
épargné sa peine, hein ! Là, c’est le roi du curry vert.

-- Honneur aux femmes.

-- Ça sent ultra bon, hein. J’ai l’impression d’être en Thaïlande. C’est
franchement excellent !

-- C’est cool !

-- Ah oui c’est super bon. J’espère que tu as retenu les… les astuces et les
recettes, parce que… ce que…

-- Toute la pression, là tout de suite !

-- Le féminisme est né au Danemark ! Tout le monde sait ça.

-- Exactement. Moi, je cherche (28) le petit et toi, quand tu rentres en premier,
tu me feras ça, quoi !

-- Tu vois, je t’ai fait gérer ton planning.

-- Voilà, j’ai déjà…

-- Cédric, tu as gagné…

-- Sérieux ?

-- Ma cuillère fétiche, que j’ai ramenée de Thaïlande. Ma cuillère thaï. C’est
pour toi.

-- C’est gentil !

-- C’est ton petit trophée. C’est pas grand chose mais quand même.

-- C’est super important.

-- Je fais la bise, la bouche pleine.

Allez on se retrouve la semaine prochaine pour une nouvelle rencontre. A nouveau,
il y aura de la recette, de la gourmandise et plein de petites astuces pour le
quotidien.

\vfill

\end{document}

