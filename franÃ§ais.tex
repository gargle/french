\documentclass{beamer}

\usepackage[utf8]{inputenc}
\usepackage[T1]{fontenc}
\usepackage[frenchb]{babel}

\usepackage{beamerthemesplit}
\usepackage{pgf}



\title{Français}
\subtitle{Parcours 1}
\author{Johan Laenen}
\date{\today}

\begin{document}

\frame{\titlepage}

\section[Outline]{}
\frame{\tableofcontents}



\section{Etape 2 : Entre amis}

\subsection{Borne 3 : Entre amis}

\frame
{
  \frametitle{Borne 3 : Entre amis}

  \framesubtitle{Exercice 2}
  
  \begin{itemize}
    \item Ils ont de la visite.
    \item Vous avez un accident.
    \item Elle a deux enfants.
    \item Ils ont une auto.
    \item Nous avons un appartement.
    \item Il a de la chance.
    \item Tu as une radio.
    \item J'ai 45 ans.
  \end{itemize}
}
\frame
{
  \frametitle{Borne 3 : Entre amis}

  \framesubtitle{Exercice 5}
  
  \begin{itemize}
    \item Tu as
    \item Nous aimons
    \item Tu es
    \item J'/il/elle habite
    \item Ils/elles sont
    \item Je/il/elle parle
    \item Vous êtes
    \item Ils/elles ont
    \item Vous parlez
    \item J'/il/elle aime
    \item Je suis
    \item Ils/elles habitent
  \end{itemize}
}

\section{Etape 3 : A la folie, pas du tout!}

\subsection{Exercices de révision - Etapes 1 \& 2}

\frame
{
  \frametitle{Exercices de révision - Etapes 1 \& 2}

  \framesubtitle{Exercice 1}
  
  \begin{itemize}
    \item Il a 45 ans.
    \item Elle est française.
    \item Je suis banquier.
    \item Elles ont 12 ans.
    \item Vous êtes sénégalais.
    \item Nous sommes retraitées.
  \end{itemize}
}

\frame
{
  \frametitle{Exercices de révision - Etapes 1 \& 2}

  \framesubtitle{Exercice 2}
  
  \begin{itemize}
    \item J'organise une fête le dix novembre.
    \item Le quatorze juillet, nous allon à Paris.
    \item Je ne suis pas libre le vingt et un janvier.
    \item Il est né le seize avril.
    \item le dix-sept décembre, c'est mon anniversaire.
  \end{itemize}
}

\frame
{
  \frametitle{Exercices de révision - Etapes 1 \& 2}

  \framesubtitle{Exercice 3}
  
  \begin{itemize}
    \item Non, nous n'allons pas à la fête de Martine.
    \item Non, je n'habite pas à Paris.
    \item Non, elle ne parle pas l'anglais.
    \item Non, ils ne sont pas libres le 5 avril.
    \item Non, nous n'aimons pas le football.
    \item Non, je ne suis pas musicien.
  \end{itemize}
}

\frame
{
  \frametitle{Exercices de révision - Etapes 1 \& 2}

  \framesubtitle{Exercice 4}
  
  \begin{itemize}
    \item Il s'appelle Edouard. Il a 52 ans. Il est dentiste. Il est anglais.
      Il habite à Birmingham. Il aime le tennis et la voile.
      Il parle l'anglais et l'allemand. Il est marié et il a deux enfants.
    \item Elle s'appelle Monique. Elle a 28 ans. Elle est coiffeuse.
      Elle est belge. Elle habite à Liège. Elle aime le cinéma.
      Elle est célibataire. Elle parle le français et l'italien.
    \item Il s'appelle Pierre. Il a 49 ans. Il est professeur. Il est allemand.
      Il habite a Leipzig. Il aime le ski et la lecture.
      Il parle l'allemand et l'espagnol.
  \end{itemize}
}

\subsection{Borne 5 : C'est une bonne idée!}

\frame
{
  \frametitle{Borne 5 : C'est une bonne idée!}

  \framesubtitle{Exercice 1}
  
  \begin{itemize}
    \item F - La femme de Jean-Paul s'appelle Sophie.
    \item V
    \item F - Jean-Paul est professeur de gymnastique.
    \item V
    \item I
    \item F - Luc met un pantalon brun et une chemise bleue.
    \item V
    \item F - Sophie cherche une montre pour son marie.
  \end{itemize}
}

\frame
{
  \frametitle{Borne 5 : C'est une bonne idée!}

  \framesubtitle{Exercice 3}
  
  \begin{itemize}
    \item Sophie a une idée.
    \item Ils habitent à Anvers mais ils viennent de Tokio.
    \item Vous faites un voyage.
    \item On met une cravate pour aller à la fête.
    \item Je peux vous aider?
    \item Jean-Paul est sportif.
    \item Vous avez besoin d'une montre.
    \item Dimanche, je vais au théâtre.
    \item Nous pouvons acheter une montre.
    \item Ils font un stage de tennis.
  \end{itemize}
}

\end{document}
