\documentclass[11pt, french]{report}

\usepackage[french]{babel}
\selectlanguage{french}
\usepackage[T1]{fontenc}
\usepackage[utf8]{inputenc}

\begin{document}

\chapter{18 janvier 2015 : Bien occupée}

\textit{Quand on travaille, on est très occupé. Et puis un jour, arrive
  l’heure de la retraite. C’est une des étapes de la vie qui change tout.
  Plus d’obligations, plus de contraintes, plus d’horaires. Pour certains,
  la transition n’est pas si facile. Mais pour Véronique, tout va bien. Elle
  a pris sa retraite il y a quelques années et pour elle, pas question de
  ne rien faire de tout ce temps qui lui appartient désormais! C’est ce
  qu’elle a expliqué à Christelle.}

\vfill

-- Bonjour Véronique !

-- Bonjour Christelle !

-- Alors, tu es à la retraite depuis plusieurs années, qu'est-ce que tu peux
nous dire à propos de la retraite ?

-- La retraite, c'est une nouvelle vie où l'on pense un peu plus à soi, et
aussi aux autres personnes. En faisant par exemple des activités physiques
comme des randonnées, de la gymnastique, du jardinage, des sorties, voilà,
du bénévolat \footnote{\textbf{Le bénévolat} : Le bénévolat est une activité
  non rémunérée qui a pour but de rendre service. On dit qu'on
  \textit{fait du bénévolat}, ou \textit{du travail bénévole}. On est
  bénévole dans un association par example.} aussi.

-- Quel genre de bénévolat ?

-- Et bien par exemple, pour ma part, je suis bénévole en tant que réserviste
de la Sécurité Civile.

-- D'accord, qu'est-ce que c'est ?

-- Les bénévoles sont des personnes réservistes, donc pour \ldots\ en cas de
risques majeurs, c'est-à-dire en cas d'inondation, de tremblement de
terre \ldots\ 

-- De feux ?

-- De feux de forêt ! Donc nous venons au secours de la population sous \ldots
C'est dirigé par le \ldots\ par le maire de la commune
\footnote{\textbf{Le maire de la commune} : les communes sont les villes
  françaises, petites ou grandes. Elles sont administrées par un Maire et
  son conseil municipal.}, voilà. Et je m'occupe également de mes
petites-filles de temps en temps, pour soulager
\footnote{\textbf{Soulager} : rendre service, par exemple ici en prenant en
  charge les petits-enfants, ce qui libère du temps pour leurs parents et
  donc les soulage.} mes enfants, surtout pendant les vacances scolaires
\footnote{\textbf{Pendant les vacances scolaires} : les enfants ont
  davantage de vacances que les parents. Donc il faut trouver comment les
  occuper quand leurs parents ne sont pas en vacances. Certains vont au
  centre aéré, avec des animateurs. D'autres sont gardés par leurs
  grands-parents par exemple.}. Donc bah à la maison où on fait des
activités, des loisirs créatifs. On emmène aussi les enfants aux sports
d'hiver.

-- À la neige ?

-- Ben voilà oui, faire du ski. Aussi, nous faisons des petits voyages. Alors
en groupe, avec des amis. Bon, ce sont des petits voyages de \ldots\ d'une
semaine, quelques jours, ou alors des sorties aussi culturelles d'une journée.

-- Tu ne pars jamais une longue période ?

-- Pendants les vancances d'été, voilà, trois \ldots\ deux \ldots\ quinze
jours\footnote{\textbf{Quinze jours} : les Français utilisent très souvent
  cette expression, à la place de << deux semaine s>>. Par exemple :
  \textit{J'ai 15 jours de vacances. / Nous avons passé 15 jours à la mer
    cet été. / On se voit dans 15 jours}.}, trois semaines maximum.

-- D'accord, pas très longtemps.

-- Ben c'est déjà \ldots\ c'est déjà bien.

-- Oui ! Donc tu te \ldots\ Tu t'ennuies \footnote{\textbf{Tu T'ennuies} : du
  verbe << s'ennuyer >> qui signifie qu'on ne trouve rien à faire pour
  s'occuper.} vraiment jamais alors ?

-- Non, je m'occupe, je suis active, et je ne m'ennuies pas. Je m'occupe aussi
beaucoup de la maison, entre autres, bah, les courses, l'entretien
\footnote{\textbf{L'entretien} : de la maison (le nettoyage \ldots\ )}.
Et puis je vais rendre visite \footnote{\textbf{rendre visite} : on peut
  dire aussi : \textit{Je vais voir des amis}. (Mais en français, on ne dit pas
  qu'on visite quelqu'un, contrairement à l'anglais. \textit{On rend visite
    à quelqu'un}.)} aussi quelquefois à des amis, on fait des petites
réunions \footnote{\textbf{Réunions} : ici, ce sont des rendez-vous entre
  amis. On pourrait dire aussi: \textit{On se réunit entre nous}.} entre nous,
c'est bien sympathique. Et puis c'est vrai que bon, le temps passe \ldots\
passe vraiment vite quand on est bien occupé, et qu'on se fait plaisir, et qu'on
fait plaisir aux autres.

-- Eh oui, il faut en profiter !

-- Tout à fait !

-- Eh bien merci, Véronique !

-- Bien merci à toi !

\end{document}
