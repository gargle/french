\documentclass[11pt, french]{report}

\usepackage[french]{babel}
\selectlanguage{french}
\usepackage[T1]{fontenc}
\usepackage[utf8]{inputenc}
\usepackage{hyperref}

\begin{document}

\chapter{22 décembre 2008~: Le repas de Noël}

\url{https://francebienvenue1.wordpress.com/2008/12/}

\vfill

\textit{Un repas traditionnel\ldots\ Miam miam\ldots\ Plein des bonnes
choses\ldots\ \\
Hé, les gourmands, c'est par ici~!\\
Vous saurez tout sur le repas traditionnel de Noël en Provence grâce à Lola
qui était dans la cuisine de sa maman juste avant le 25 décembre.} 

\vfill

L~: Alors mamam, qu'est-ce que tu nous prépares de bon pour Noël~?

C~: (Ben)\footnote{<<~Ah~>> et <<~Ben~>> sont des interjections.  ``Ben'' vient
  de ``Eh bien'', mais quand on parle familièrement, on ne prend pas le temps
  de bien prononcer ces 2 mots.} cette année on va faire un repas traditionnel.

L~: Traditionnel, c'est-à-dire~? Une dinde~? Du foie gras~?

C~: Non pas tout à fait.  En entrée, on va faire des huîtres, du saumon et
du foie gras puis en plat principal, on va faire un chapon.

L~: Ouais\footnote{<< Ouias >> est un mot familier pour dire <<~oui~>>.  Il faut
  essayer de ne pas l'utiliser tout le temps~!}\ldots

C~: Et puis\ldots

L~: Donc pas de dinde alors~?

C~: Non, pas de dinde cette année.

L~: Bon, et heu\ldots\ Et comme dessert~?

C~: (Ah ben) cette année, je vais faire un dessert qui rappelle la Provence, ce
sont les 13 desserts, et en fait, c'est une tradition qui fait qu'il faut manger
de chacun des 13 pour avoir une bonne année qui vient.

L~: D'accord, c'est donc une nouveauté cette année.

C~: Qui, oui

L~: On ne l’avait pas fait avant.

C~: Non, non. Alors, en fait\ldots

L~: Mais c’est comment~?

C~: Les 13 desserts en fait, il y a d’abord les noisettes, les figues sèches,
puis des amandes et des raisins.

L~: Mm\ldots

C~: Et puis après, on fait 2 sortes de nougats : le nougat blanc aux noisettes,
avec des pignons de pin et de pistaches, et le nougat noir qui est au miel
fondu, cuit dans des amandes. Puis après, on fait un petit peu de pâte de
coing, puis on met des fruits confits, puis on met ce qu’on appelle des
raisins blancs d’hiver.

L~: Donc là, tu es en train de me décrire les 13 desserts~?

C~: Voilà, j’ai dit les 13 qu’il faut absolument mettre sur la table.

L~: D’accord.

C~: Et puis après le raisin blanc, si on n’en a pas, on peut mettre d’autres
fruits frais comme des pommes ou des poires, puis après on met ce qu’on
appelle le melon de Noël qui est en fait un melon vert, puis on met aussi des
oranges et des clémentines, et puis les calissons d’Aix. Ou bien, si on n’en
a pas, on peut mettre des dattes confites, et puis un gâteau que l’on appelle
la pompe à huile qui est en fait une fougasse briochée, que je vais faire
aujourd’hui.

L~: D’accord.

C~: Comme ça, elle sera faite~!

L~: Ça va alors. Moi je croyais que ce serait beaucoup trop mais en fait c’est
\footnote{<<~C'est~>> est faux grammaticalement. A l'écrit, il faudrait dire~:
  <<~ce sont~>>. Mais c'est très fréquent à l'oral~!} des petits desserts, ce
sera mangeable.

C~: Voilà.

L~: Je pensais 13 desserts, ça fait gros quand même\ldots\ Mais non.

C~: Non, en fait, on prend une bouchée de chacun.

L~: Ce sont des tout petits desserts~?

C~: Voilà.

L~: Chouette\footnote{<<~Chouette~>> est un mot familier. Synonyme de super.}
alors~!

C~: Alors pour présenter~:\\
ou bien on fait un très grand plat où on les met tous (dedans),\\
ou bien on les met chacun dans une assiette, une petite chose de tous dans
chaque assiette.\\
Je vais voir un petit peu comment je vais faire.

L~: Hé ben, on va voir ce que ça donne~! Merci~!

\end{document}
