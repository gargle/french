\documentclass[11pt, french]{report}

\usepackage[french]{babel}
\selectlanguage{french}
\usepackage[T1]{fontenc}
\usepackage[utf8]{inputenc}

\begin{document}

\chapter{26 janvier 2015 : La nouvelle année}

\textit{Ça y est, nous sommes en 2015 ! C’est le moment de faire le bilan
  de l’année 2014 et de se fixer des objectifs à atteindre pour la nouvelle
  année. Cette fois-ci, nous sommes allés demander à nos camarades (de l’IUT)
  leurs impressions et leurs projets. Attention mesdames et messieurs ,
  voici Mickaël, Marie, Sarah, Nicolas et Camille !}

\vfill

-- Bonne année, Mickael !

-- Bonne année, Shainesse !

-- Alors, ça s'est bien passé, les fêtes \footnote{\textbf{les fêtes} : employé
  seul comme ici, ce mot désigne toujours les fêtes de fin d'année, la période
  autour de Noël et du Jour de l'An.} ?

-- Nickel \footnote{\textbf{Nickel} : expression familière et orale qui
  signifie que c'était très bien, parfait.}, et toi ?

-- Oui, ça s'est bien passé. Alors, c'est quoi
\footnote{\textbf{C'est quoi, tes résolutions ?} : style très oral et très
  courant. La manière tout à fait correcte serait de demander : \textit{Quelles
  sont tes résolutions ?} (Mais dans une langue plus soutenue, plus
<< surveillée >> en quelque sorte.)}, tes nouvelles résolutions pour l'année 2015 ?

-- Alors, pour 2015, je veux arrêter de boire, arrêter de fumer. Je veux
apprendre à jouer au piano \footnote{\textbf{jouer au piano} : ça ne se dit
  pas ! On dit : \textbf{jouer du piano}, \textit{jouer d'un instrument de
    musique}. En revanche, pour les sports ou les jeux, on emploie << au >>,
avec des noms masculins, ou pluriel : \textit{jouer au foot, au basket, au
tennis. / Jouer aux échecs, jouer aux cartes}.}. J'en ai toujours rêvé et je
pense que cette année, c'est la bonne \footnote{\textbf{C'est la bonne !} :
cela signifie que ce sera enfin l'année où ce rêve ou ce projet va se
réaliser.} !

-- Ah, c'est génial \footnote{\textbf{C'est génial} : c'est super. (familier)}
, le piano ! Bah écoute, j'espère queça va bien se passer et que tu vas
devenir le nouveau Beethoven \ldots\ J'espère qu'il jouait au piano
\footmark[4] ! Et \ldots\ voilà ! Ciao ciao !
\footnote{\textbf{Ciao, ciao} : les Français aiment bien utiliser ces mots
  d'Italien ! Notamment dans le sud-est de la France. On entend souvent
<< Tcha Tchao >> en fait, dit très vite.}

\end{document}
